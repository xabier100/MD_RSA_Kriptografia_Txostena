\documentclass[12pt]{article}
\usepackage[a4paper]{geometry}
\usepackage[myheadings]{fullpage}
\usepackage{fancyhdr}
\usepackage{lastpage}
\usepackage{graphicx, wrapfig, subcaption, setspace, booktabs}
\usepackage[T1]{fontenc}
\usepackage[font=small, labelfont=bf]{caption}
\usepackage{fourier}
\usepackage[protrusion=true, expansion=true]{microtype}
\usepackage[basque]{babel}
\usepackage{sectsty}
\usepackage{url, lipsum}
\usepackage{graphicx}
\usepackage[utf8]{inputenc}
\usepackage{courier}
\usepackage{ulem}
\usepackage{spverbatim}
\usepackage{multirow} %para las tablas
\usepackage{color}
\usepackage{graphicx}
\usepackage{epsfig}
\usepackage{multirow}
\usepackage{colortbl}
\usepackage{xcolor}
\usepackage{float}
\usepackage{fancyvrb}
\usepackage{bera}
\usepackage{adjustbox}
\usepackage{amsmath}


\usepackage{listings} %For code in appendix
\lstset
{ %Formatting for code in appendix
    language=Matlab,
    basicstyle=\footnotesize,
    numbers=left,
    stepnumber=1,
    showstringspaces=false,
    tabsize=4,
    breaklines=true,
    breakatwhitespace=false,
}

\newcommand{\HRule}[1]{\rule{\linewidth}{#1}}
\onehalfspacing
\setcounter{tocdepth}{5}
\setcounter{secnumdepth}{5}

%-------------------------------------------------------------------------------
% HEADER & FOOTER
%-------------------------------------------------------------------------------
\pagestyle{fancy}
\fancyhf{}
\setlength\headheight{15pt}
\fancyhead[R]{\MakeUppercase{matematika diskretua}}
\fancyfoot[R]{\thepage}
%-------------------------------------------------------------------------------
% TITLE PAGE
%-------------------------------------------------------------------------------

\begin{document}

\title{ \normalsize \textsc{Matematika Diskretua}
		\\ [2.0cm]
		\HRule{2pt} \\
		\LARGE \textbf{RSA Kriptografia Txostena}
		\HRule{2pt} \\ [0.5cm] \bigskip
		\normalsize \today \vspace*{1\baselineskip}}

\date{}

\author{
        \emph{Egilea:} \\ \\
		Xabier Garrote }

\maketitle
\newpage
\tableofcontents
\newpage


%--------------------------------------------------------------------% Documentuaren atalak ezberdinak hemendik aurrera ---------------------------------------------------------------------
\section{Sarrera}
\section{Inplementatutako funtzioen kodea}

\subsection{zkh 1.bertsioa}
\begin{verbatim}
    # Bi zenbaki oso emanik, a eta b, 
    # zatitzaile komunetako handiena 
    # kalkulatuko duen funtzioaren 
    # inplementazioa Euklidesen algoritmoan 
    # oinarrituz:
    zkh <- function(a,b){ 
        c<-a;
        d<-b;
        while (d != 0) {
            r=mod(c,d);
            c=d;
            d=r;
        }
        return(c)  
    }
\end{verbatim}
\subsubsection{Deiak funtzioari}
\begin{verbatim}
    > zkh(2689,4001)
    [1] 1
    > zkh(1369,2597)
    [1] 1
    > zkh(3.2,4)
    Error in mod(c, d) : 
    Arguments 'n', 'm' must be integers or vectors of integers. 
    > zkh("a","b")
     Error in mod(c, d) : is.numeric(n) is not TRUE 
\end{verbatim}

\newpage

\subsection{zkh 2.bertsioa}
\begin{verbatim}
    # Parametro moduan jasotako a,b Euklidesen algoritmoan oinarrituz
    # zatitzaile komunetako handiena itzuliko du
    # Parametro moduan zenbaki errealak pasaz gero, errore
    # mezua erakutsiz, karaktereak pasaz gero, baita.
    zkh <- function(a,b){ 
        if(is.character(a)==FALSE & is.character(b)==FALSE){
            c=abs(a)
            d=abs(b)
            if(isNatural(c)&isNatural(d)){
                c<-a;
                d<-b;
                while (d != 0) {
                    r=mod(c,d);
                    c=d;
                    d=r;
                }
                return(c)
            }
            else{
                print('a eta b parametroak zenbaki osoak izan behar dute.')
            }
        }
        else{  
            print('a eta b ezin dira karaktereak izan.')
        }
    }
\end{verbatim}

\newpage

\subsubsection{Deiak funtzioari}
\begin{verbatim}
    > zkh(231,1820)
    [1] 7
    > zkh(1369,2597)
    [1] 1
    > zkh(3.2,4)
    [1] "a eta b parametroak zenbaki osoak izan behar dute." 
    > zkh(5,'b')
    [1] "a eta b ezin dira karaktereak izan."
\end{verbatim}

\newpage
\subsection{lehen\_erlatibo\_txiki}
\begin{verbatim}
    # Parametro moduan funtzioari pasatako m-rekin lehen erlatiboa 
    # den zenbaki oso positiborik txikiena itzuliko du
    # RSA_gakoak funtzioa inplementatzeko erabiliko dugu
    lehen_erlatibo_txiki <- function(m){
        zk = 2
        while (GCD(m, zk) != 1){
            zk = zk + 1;
        }
        return(zk);
    }
\end{verbatim}

\subsubsection{Deiak funtzioari}
\begin{verbatim}
    > lehen_erlatibo_txiki(13797)
    [1] 2
    > # 2
    > lehen_erlatibo_txiki(16974)
    [1] 5
    > # 5
    > lehen_erlatibo_txiki(56970)
    [1] 7
    > # 7
    > lehen_erlatibo_txiki(1000000000000000)
    [1] 3
\end{verbatim}
    
\newpage

\subsection{RSA\_gakoak}
\begin{verbatim}
    # Bi zenbaki lehen emanik, p eta q, n, r eta s kalkulatuko 
    # ditu, eta irteera estandarretik inprimatu. r txikia kalkulatuko
    # da.
    RSA_gakoak <- function(p,q){
        n = p * q;
        m = (p - 1) * (q - 1);
        r = lehen_erlatibo_txiki(m);
        s = modinv(r, m);
        cat("n = ", n, ", r = ", r, ", s = ", s); 
    }
\end{verbatim}
\subsubsection{Deiak funtzioari}
\begin{verbatim}
    > RSA_gakoak(5,17)
    n =  85 , r =  3 , s =  43
    > RSA_gakoak(17,23)
    n =  391 , r =  3 , s =  235
    > RSA_gakoak(97,101)
    n =  9797 , r =  7 , s =  2743
    > RSA_gakoak(307,397)
    n =  121879 , r =  5 , s =  96941
\end{verbatim}

\newpage

\subsection{kodetu}
\begin{verbatim}
    # Parametro moduan txt karaktere string-a jaso eta dagozkion  
    # ASCII kodeak itzultzen ditu.
    kodetu <- function(txt){
        return(strtoi(charToRaw(txt), 16L));
    }
\end{verbatim}

\subsubsection{Deiak funtzioari}
\begin{verbatim}
    # Ondoren, deskodetu funtzioarekin konprobatuko dugu  
    # ondo kodetu direla
    > testu1<-"kaixo"
    > kodebektore1<-kodetu(testu1)
    > kodebektore1
    [1] 107  97 105 120 111
    > testu2<-"KAIXO"
    > kodebektore2<-kodetu(testu2)
    > kodebektore2
    [1] 75 65 73 88 79
    > testu3<-"Zer moduz?"
    > kodebektore3<-kodetu(testu3)
    > kodebektore3
     [1]  90 101 114  32 109 111 100 117 122  63
\end{verbatim}

\newpage

\subsection{deskodetu}
\begin{verbatim}
    # Parametro moduan ASCII kodeak jaso eta dagokion txt karaktere 
    # string-a itzultzen duen funtzioa.
    deskodetu <- function(kodetxt){
        return(rawToChar(as.raw(kodetxt)));
    }   
\end{verbatim}

\subsubsection{Deiak Funtzioari}
\begin{verbatim}
    # Kodetu funtzioarekin kodetutatko bektoreak 
    # deskodetuko ditugu
    > txt1<-deskodetu(kodebektore1)
    > txt1
    [1] "kaixo"
    > txt2<-deskodetu(kodebektore2)
    > txt2
    [1] "KAIXO"
    > txt3<-deskodetu(kodebektore3)
    > txt3
    [1] "Zer moduz?"
\end{verbatim}

\newpage

\subsection{zifratu}
\begin{verbatim}
    # Parametro moduan ASCII kodeez osatutako bektore bat jaso eta 
    # dagokion bektore zifratua itzuliko du
    zifratu <- function(kodebektorea,r,n){
        luzeera = length(kodebektorea);
        for (i in 1:luzeera)
        {
            kodebektorea[i] = modpower(kodebektorea[i], r, n);
        }
        return(kodebektorea);
    }
\end{verbatim}

\subsubsection{Deiak funtzioari}
\begin{verbatim}
    > # Erabiliko ditugun gakoak:
    > # n=9797, r=7, s=2743
    > bektorezifratu1<-zifratu(kodebektore1,7,9797)
    > bektorezifratu1
    [1] 2792 5432 4668 4973 7969
    > bektorezifratu2<-zifratu(kodebektore2,7,9797)
    > bektorezifratu2
    [1] 7976 4764 2565 8540 4974
    > bektorezifratu3<-zifratu(kodebektore3,7,9797)
    > bektorezifratu3
     [1]  375 2222 7721 3675  493 7969 6261 8564 4122 4604
\end{verbatim}

\newpage

\subsection{deszifratu}
\begin{verbatim}
    # Parametro moduan bektore zifratu bat jaso eta dagozkion 
    # ASCII kodeak itzuliko du
    deszifratu <- function(bektorezifratu,s,n){
        luzeera = length(bektorezifratu);
        for (i in 1:luzeera)
        {
            bektorezifratu[i] = modpower(bektorezifratu[i], s, n);
        }
      return(bektorezifratu);
    }
\end{verbatim}

\subsubsection{Deiak funtzioari}
\begin{verbatim}
    > # RSA_gakoak(97,101) zenbaki lehenetatik lortutako gakoekin:
    > # n=9797, r=7, s=2743
    > n<-9797
    > r<-7
    > s<-2743
    > testuberria<-"ea hau ondo doan..."
    > testuberri_zifratua <- zifratu(kodetu(testuberria),r,n)
    > testuberri_zifratua
     [1] 2222 5432 3675 1177 5432 8564 3675 7969 9305 6261
    [11] 7969 3675 6261 7969 5432 9305 3037 3037 3037
    > testuberri_errekuperatua <- deskodetu(deszifratu(testuberri_zifratua,s,n))
    > testuberri_errekuperatua
    [1] "ea hau ondo doan..."
\end{verbatim}
\section{Bibliografia}



\section{Iritzi pertsonala}




\end {document}